\documentclass[oneside,ebook]{memoir}
\chapterstyle{thatcher}

\usepackage{polyglossia}
\usepackage{tabularx}
\usepackage{enumitem}
\usepackage[svgnames, dvipsnames]{xcolor}
\definecolor{myBlack}{cmyk}{0,0,0,1}
%% Start of CMYK Black options
\makeatletter
\newcommand{\globalcolor}[1]{%
  \color{#1}\global\let\default@color\current@color
}
\makeatother
%% End of CMYK Black options

\setmainlanguage{english}
\setotherlanguage{arabic}

\newfontfamily\arabicfont[
Script=Arabic,
Numbers=Proportional,
Scale=1.4%,
]{Scheherazade}

\setmainfont[
Numbers=OldStyle,
BoldFont={AGaramondPro-Bold.otf}, 
ItalicFont={AGaramondPro-Italic.otf},
BoldItalicFont={AGaramondPro-BoldItalic.otf}
]{AGaramondPro-Regular.otf}

\begin{document}
\chapter*{Tr\kern 1ptansliteration Key}
\vspace{-2.3em}
\thispagestyle{empty}

\begin{table}[!h]
{\def\arraystretch{1.2}\tabcolsep=5pt
\begin{tabular}{@{}cl@{}}
\textarabic{ء} & {\large ʾ}\textsuperscript{(1)}\\
\textarabic{ا} & {\large ā, a}\\
\textarabic{ب} & {\large b}\\
\textarabic{ت} & {\large t}\\
\textarabic{ث} & {\large th}\textsuperscript{(2)}\\
\textarabic{ج} & {\large j}\\
\textarabic{ح} & {\large ḥ}\textsuperscript{(3)}\\
\textarabic{خ} & {\large kh}\textsuperscript{(4)}\\
\textarabic{د} & {\large d}\\
\textarabic{ذ} & {\large dh}\textsuperscript{(5)}\\
\end{tabular}}
\hfill
{\def\arraystretch{1.2}\tabcolsep=5pt
	\begin{tabular}{@{}cl@{}}
\textarabic{ر} & {\large r}\textsuperscript{(6)}\\
\textarabic{ز} & {\large z}\\
\textarabic{س} & {\large s}\\
\textarabic{ش} & {\large sh}\\
\textarabic{ص} & {\large ṣ}\textsuperscript{(7)}\\
\textarabic{ض} & {\large ḍ}\textsuperscript{(8)}\\
\textarabic{ط} & {\large ṭ}\textsuperscript{(9)}\\
\textarabic{ظ} & {\large ẓ}\textsuperscript{(10)}\\
\textarabic{ع} & {\large ʿ}\textsuperscript{(11)}\\
\textarabic{غ} & {\large gh}\textsuperscript{(12)}\\
\end{tabular}}
\hfill
{\def\arraystretch{1.2}\tabcolsep=5pt
\begin{tabular}{@{}cl@{}}
\textarabic{ف} & {\large f}\\
\textarabic{ق} & {\large q}\textsuperscript{(13)}\\
\textarabic{ك} & {\large k}\\
\textarabic{ل} & {\large l}\\
\textarabic{م} & {\large m}\\
\textarabic{ن} & {\large n}\\
\textarabic{ه} & {\large h}\textsuperscript{(14)}\\
\textarabic{و} & {\large ū, u, w}\\
\textarabic{ي} & {\large ī, i, y}\\
\textarabic{} & {\large }\\
\end{tabular}}
\end{table}
\vspace{-0.95em}
\begin{small}
\begin{enumerate}[leftmargin=*]
\setlength\itemsep{-0.2em}
\item A distinctive glottal stop made at the bottom of the throat.
\item Pronounced like the \textit{th} in \textit{think}.
\item Hard \textit{h} sound made at the Adam’s apple in the middle of the throat.
\item Pronounced like \textit{ch} in Scottish \textit{loch}.
\item Pronounced like \textit{th} in \textit{this}.
\item A slightly trilled \textit{r} made behind the upper front teeth.
\item An emphatic \textit{s} pronounced behind the upper front teeth.
\item An emphatic \textit{d}-like sound made by pressing the entire tongue against the upper palate.
\item An emphatic \textit{t} sound produced behind the front teeth.
\item An emphatic \textit{th} sound, like the \textit{th} in this, made behind the front teeth.
\item A distinctive Semitic sound made in the middle of the throat and sounding to a Western ear more like a vowel than a consonant.
\item A guttural sound made at the top of the throat resembling the untrilled German and French \textit{r}.
\item A hard \textit{k} sound produced at the back of the palate.
\item This sound is like the English \textit{h} but has more body. It is made at the very bottom of the throat and pronounced at the beginning, middle, and ends of words.
\end{enumerate}
\end{small}
\end{document}