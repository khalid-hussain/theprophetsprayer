\hypertarget{analysis-of-the-narrations-regarding-the-saying-of-ux101mux12bn-by-the-imam-and-the-congregation}{%
\chapter{Analysis of the narrations regarding the saying of āmīn by the
Imam and the
Congregation}\label{analysis-of-the-narrations-regarding-the-saying-of-ux101mux12bn-by-the-imam-and-the-congregation}}

\chaptermark{The saying of āmīn by the Imam and the Congregation}

From: \emph{Silsilah al-Aḥādīth al-Ḍaʿīfah} (\#951--\#952) by Shaykh
al-Albānī.

\begin{mdframed}[style=narration, frametitle={Narration \#951}]
When he said \textit{āmīn}, those behind him would say \textit{āmīn}, such that there was a lot of noise in the mosque.
\end{mdframed}

There is \textsc{no basis} for the \emph{ḥadīth} with this wording as
far as we know. Ibn Ḥajr said in \emph{al-Talkhīṣ al-Ḥabīr} (p.~90), ``I
do not find it with this wording, but its meaning is related by Ibn
Mājah in the \emph{ḥadīth} of Bishr bin Rāfiʿ (i.e. \textit{ḥadīth} \#952).''
He then said,

\begin{quote}
\textcolor{MidnightBlue}{\textsc{Note}: Ibn Ṣalāḥ said about this \emph{ḥadīth} in his book
\emph{al-Wasīṭ}, ``This \emph{ḥadīth} was quoted by al-Ghazzālī with
this wording just as Imam al-Ḥaramayn cited it in his book
\emph{al-Nihāyah}, which is not authentic as being \emph{marfūʿ}.''
Al-Shāfiʿī transmitted it from the \emph{ḥadīth} of ʿAṭā who said, ``I
used to hear the Imams Ibn Zubayr and those who came after him all say
\emph{Āmīn} (loudly) until the mosque trembled.'' Al-Nawawī said
something similar and he added that this is incorrect from them both;
and it seems as he and Ibn Ṣalāḥ both intended this wording of the
\emph{ḥadīth} to be correct. However, the context provided by Ibn Mājah
provides some meaning (to the \emph{ḥadīth}) we mentioned earlier.}
\end{quote}

\textcolor{MidnightBlue}{The statements which have preceded establish that the context provided
by Ibn Mājah is the complete meaning and not just some part of it.
Therefore, if one thinks about the context, then either all of the
meaning is contained in it or just a part. If we take just some parts of the \emph{ḥadīth}, it shows only the Imam
should say \emph{Āmīn} loudly, which is clear from the text. And if we
take the entire context and meaning, it provides evidence that the
congregation also said \emph{Āmīn} loudly with the Imam as the wording
of the \emph{ḥadīth} says, ``\ldots{}and the mosque would shake with
it.'' Therefore, it is possible that the reason for the echoing of the
\emph{Āmīn} was due to the Messenger of Allāh \pbuh saying it as the
\emph{ḥadīth} is conclusive regarding this and it is also possible this
was due to the congregation's saying \emph{Āmīn} coinciding with his.
And this is the wording of the \emph{ḥadīth} of Ibn Mājah.}

\begin{mdframed}[style=narration, frametitle={Narration \#952}]
When he recited “Not of those who received Your anger, nor of those who go astray,” he said \textit{āmīn}, such that those close to him in the first row could hear (and the mosque trembled with it).
\end{mdframed}

\emph{Ḍaʿīf} (\textsc{Weak}). Related by Ibn Mājah (1/281) and Abū Dāwūd
without the addition (1/148), both via:

Bishr bin Rāfiʿ from Abū ʿAbd Allāh, son of the maternal uncle of (i.e. cousin of) Abū Hurayrah \mabpwhim,
from Abū Hurayrah \mabpwhim from the Prophet \pbuh.

\textcolor{MidnightBlue}{This chain is weak. Al-Ḥafiẓ
al-Zarʿah ibn al-ʿIraqī's saying, “Its chain is good,” (\textit{Ṭarḥ al-Tathrīb} [2/268]) is not good and will be clarified with proceeding statements.}
%It makes clear what is brought forward from the \emph{ḥadīth}.}

Ibn Ḥajar said in \emph{al-Talkhīṣ} (p.~90), ``Bishr bin Rāfiʿ is weak;
the cousin of Abū Hurayrah \mabpwhim has been said to be unknown, but
Ibn Ḥibbān has declared him reliable.''

Al-Būṣayrī said in \emph{al-Zawāʾid} (56/1), ``This is a weak
\emph{isnād}; Abū ʿAbd Allāh's condition is not known; Bishr was
declared weak by Aḥmad, and Ibn Ḥibbān said, `He narrated
fabrications.'\,''

\textcolor{MidnightBlue}{What Ibn Ḥibbān said was good when he concluded (1/179), ``\ldots{}as if
  he did this intentionally.'' It was from the shortcomings of al-Shawkānī
  when he said about this \emph{ḥadīth}---after Majd ibn Taymīyyah
  transmits it with the wording of Abū Dāwūd and Ibn Mājah
  (2/188)---``Al-Dāraquṭnī also transmitted it and he said, `The chain is
  \emph{ḥasan}' and al-Ḥākim said, `It is \emph{ṣaḥīḥ} according to the
  conditions of al-Bukhārī and Muslim.' And al-Bayhaqī said, `It is
  \emph{ḥasan ṣaḥīḥ}.'\,''}
  
\textcolor{MidnightBlue}{All of them transmitted the first part of the
\emph{ḥadīth} with the wording, ``When he finished reciting the \emph{Umm al-Qurʾān},
he raised his voice and said \emph{Āmīn},'' and it does not contain the words,
``\ldots{}those close to him in the first row could hear it.'' The
wording of this narration is not in the narration of Ibn Mājah; that the
\emph{Āmīn} said by the congregation coincided with his
\pbuh \emph{Āmīn} and that the mosque would shake. This shows the
difference between the narrations and therefore it is not permissible to
ascribe to the first narration from the two except excluding it from the
other narrations is apparent. Furthermore, the wording in the narration
is also weak as the narrator Isḥāq bin Ibrāhīm bin al-ʿAlā
al-Zabīdī---who is in all of the chains better known as Ibn Zabrīq---is
weak. Abū Ḥātim said, ``Shaykh and there is no harm in him.'' Ibn Maʿīn
praised him with good, while al-Nasāʾī said, ``He is not trustworthy,''
and Muḥammad bin ʿAwf said, ``I doubt that Isḥāq bin Zabrīq used to
lie.''}

\textcolor{MidnightBlue}{However, the wording of this narration is correct and it has a
  supporting \emph{ḥadīth} from Wail bin Ḥujr, with an authentic chain. As
  for the wording of the first narration, then I do not know anything that
  supports it from the Sunnah except what al-Shāfiʿī transmits in his
  \emph{Musnad} (1/76):}
  
  \begin{quote}
    \textcolor{MidnightBlue}{Muslim bin Khālid informed us from Ibn Jurayj from ʿAṭā who said, ``I
    used to hear the Imams\ldots{}'' and he mentioned Ibn Zubayr and those
    who came after him that ``when they would say \emph{Āmīn}, those behind
    them would also say \emph{Āmīn} to the extent that the mosque shook
    (i.e.~trembled).''}
  \end{quote}

\textcolor{MidnightBlue}{Al-Ḥāfiẓ Ibn Ḥajr remained silent about the \emph{ḥadīth} as has
  preceded and it has two defects. The first: The weakness of Muslim bin
  Khālid and he is known as al-Zanjī. Al-Ḥāfiẓ Ibn Ḥajr said he was
  truthful but made many mistakes. The second: The \emph{ʿanʿanah} of Ibn
  Jurayj, who was a \emph{mudallis} and it is possible that he took the
  narration from Khālid bin Abī Anuwf who transmitted from ʿAṭā with the
  wording, ``I met 200 Companions of the Messenger of Allah \pbuh in this
  mosque (i.e.~Masjid al-Ḥarām) that when the Imam would say, `(INSERT
  AYAH),' they would raise their voices and say \emph{Āmīn},'' and in
  another narration, ``I heard the echo of their \emph{Āmīn}.''
  Transmitted by Ibn Ḥibbān in \emph{al-Thiqāt} (2/74), al-Bayhaqī (2/59)
  who has another narration. This Khālid in the chain, Ibn Abī Ḥātim has
  an entry for him in (1/2/355--356) {[}Translator's Note: in
  \emph{al-Jarḥ wa al-Taʿdīl}{]} and he does not mention anything by the
  way of disparagement or praise. Ibn Ḥibbān cites this report under his
  biographical entry in his \emph{al-Thiqāt}. However, Ibn Ḥibbān is
  well-known to be lenient in grading narrators to be trustworthy.
  Therefore, it is for this reason that I am not satisfied regarding the
  authenticity of this narration. This is because if Ibn Jurayj did take
  this from him, then it is only one chain of transmission and if this is
  not the case, then we do not know who Ibn Jurayj took it from. It seems
  that Imam al-Shāfiʿī himself was also not satisfied with the
  authenticity of this narration and holds a different view. He says in
  \emph{al-Umm} (1/95), ``So when the Imam finished reciting \emph{Umm
    al-Qurʾān}, he should say \emph{Āmīn} while raising his voice so that
  the congregation can follow him in that. When he says it, they say it
  within themselves but I do not like them saying it loudly.'' If this
  report was authentic from the Companions according to al-Shāfiʿī, he
  would not have liked to have opposed their action, \emph{in shā Allāh}.
  Therefore, the most correct view in this issue is the view of
  al-Shāfiʿī; that only the Imam should say \emph{Āmīn} loudly and not the
  congregation. Allāh knows best.} % Imam Al-Shāfiʿī opinion Amin.

This \emph{ḥadīth} only gives a part of the meaning of the one before
it, i.e.~the saying of \emph{āmīn} by the Imam alone. As for the
\emph{āmīn} of those behind, this could be the reason for the phrase
``the mosque trembled with it (the sound),'' but the \emph{ḥadīth}
literally implies that the \emph{āmīn} of the Prophet \pbuh was the
reason for this.

\begin{mdframed}[style=narration, frametitle={Narration}]
When he finished reciting the Mother of the Quran, he raised his voice and said \textit{āmīn}.
\end{mdframed}

\emph{Ḍaʿīf} (\textsc{Weak}). Related by al-Dāraquṭnī, al-Ḥākim, and
al-Bayhaqī.

All the above sources contain Isḥāq bin Ibrāhīm bin al-ʿAlā al-Zubaydī,
also known as Ibn Zibrīq, who is weak: Abū Ḥātim said, ``An old man, no
harm in him;'' Ibn Maʿīn described him in good terms; al-Nasāʾī said,
``Not reliable;'' Muḥammad bin ʿAwf said, ``I have no doubt that Isḥāq
bin Zibrīq used to lie.'' However, this wording is correct in meaning,
for it has a supporting \emph{ḥadīth} of Wāʾil bin Ḥajar with a
\emph{ṣaḥīḥ sanad}.

(Since the text of this \emph{ḥadīth} does not imply the \emph{āmīn} of
the congregation at all, it is incorrect to regard it as another version
of \emph{ḥadīth} no. 2, as al-Shawkānī did.)

The only support for no. 1 is what al-Shāfiʿī related in his
\emph{Musnad} (1/76) via Muslim bin Khālid from Ibn Jurayj from ʿAtā,
who said:

\begin{mdframed}[style=narration, frametitle={Narration}]
I used to hear the Imams: Ibn al-Zubayr and others after him would say \textit{āmīn}, and those behind would say \textit{āmīn}, until the mosque echoed.
\end{mdframed}

This has two defects:

\begin{enumerate}
\def\labelenumi{\roman{enumi}.}
\tightlist
\item
  The weakness of Muslim bin Khālid al-Zanjī; Ibn Ḥajar said, ``He was
  truthful, but made many errors.''
\item
  The \emph{ʿanʿanah} of Ibn Jurayj, who was a \emph{mudallis}; perhaps
  he actually took it from Khālid bin Abī Anūf, who narrated it from
  ʿAtā as follows:
\end{enumerate}

\begin{mdframed}[style=narration, frametitle={Narration}]
I came across two hundred Companions of the Messenger of Allāh \pbuh in this mosque (i.e. Masjid al-Ḥarām, Makkah): when the Imam had said “Nor of those who go astray,” they raised their voices in \textit{āmīn} (in one narration: I heard the thundering sound of their \textit{āmīn}).
\end{mdframed}

Related by al-Bayhaqī (2/59) and Ibn Ḥibbān in \emph{al-Thiqāt} (2/74);
the alternative narration is from the former.

This Khālid was described by Ibn Abī Ḥātim (1/2/355--366), but he did
not include any authentication or disparagement. Ibn Ḥibbān included him
among the reliable narrators, but Ibn Ḥibbān is well-known to be far
from rigorous in such cases, so I am not satisfied that this narration
is authentic. This is because if Ibn Jurayj indeed took it from him,
this constitutes only one debatable route; if not, we do not know from
whom Ibn Jurayj took it. It seems that Imam al-Shāfiʿī himself was not
satisfied of the authenticity of this narration, for his position is
contrary to it: he says in \emph{al-Umm} (1/95), ``So when the Imam
completes reciting the Mother of the Book, he says \emph{āmīn}, raising
his voice so that those behind may follow him: when he says it, they say
it to themselves, but I do not like them saying it aloud;'' Had the
above narration from the Companions been authentic in al-Shāfiʿī's view,
he would not have opposed their action.

Hence, the most correct opinion in this issue appears to be the
\emph{madhhab} of al-Shāfiʿī: that the Imam, but not those following,
should say \emph{āmīn} loudly. Allāh knows best.

But then, I saw that al-Bukhārī mentioned the text (only) of the
narration about Ibn al-Zubayr in his \emph{Ṣaḥīḥ} (i.e.~in
\emph{muʿallaq} form), designating it with certainty. Ibn Ḥajar said in
\emph{Fatḥ al-Bārī} (2/208), ``The connecting \emph{isnād} has been
provided by ʿAbd al-Razzāq from Ibn Jurayj from ʿAtā. He (i.e.~Ibn
Jurayj) said, `I said to him, ``Did Ibn al-Zubayr say \emph{āmīn} at the
end of the Mother of the Quran?'' He said, ``Yes, and those behind him
also said \emph{āmīn}, until the mosque echoed.'' He then said,
``Verily, \emph{āmīn} is a supplication.''\,'\,'' This is found in the
\emph{Muṣannaf} of ʿAbd al-Razzāq (2640/2), and from this route, in Ibn
Ḥazm's \emph{al-Muḥallā} (3/364).

In this narration, Ibn Jurayj has clarified that he took the narration
from ʿAtā face-to-face, so we are assured of the absence of
\emph{tadlīs}, and the narration of Ibn al-Zubayr is established firmly.
Similarly is proven from Abū Hurayrah \mabpwhim; Abū Rāfī said:

\begin{mdframed}[style=narration, frametitle={Narration}]
Abū Hurayrah used to call to prayer for Marwān bin al-Ḥakam, stipulating that the latter would not get to “Nor of those who go astray” unless he knew that Abū Hurayrah had entered the row. So when Marwān said “Nor of those who go astray,” Abū Hurayrah would say \textit{āmīn}, prolonging it. He also said, “When the \textit{āmīn} of those on the earth coincides with the \textit{āmīn} of those in the heaven, they are forgiven.”
\end{mdframed}

Related by al-Bayhaqī (2/59); its \emph{isnād} is \emph{ṣaḥīḥ}.

Hence, since nothing is established from any of the Companions other
than Abū Hurayrah and Ibn al-Zubayr \mabpwthem contrary to their
\emph{āmīn} \textsc{aloud}, this must be accepted. Presently, I know of
no narration opposing this. Allāh knows best.
